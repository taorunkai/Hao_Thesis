\documentclass[type = bachelor]{fduthesis-en}
\include{mathdef}
\fdusetup{
	style/bibresource={test.bib},
	%style/automakecover=false,
	info={
		title={\textit{N} = 1 超对称规范理论综述},
		title*={Review of \textit{N} = 1 Supersymmetric Gauge Theory},
		author={张昊},
		supervisor={Satoshi Nawata},
		department={物理系},
		major={物理学},
		studentid = {14307110144},
		keywords* = {Supersymmetric Gauge Theory, Seiberg Witten Theory}
	}
}


\begin{document}
  \tableofcontents
  \begin{abstract}
	  This paper is a summary of \textit{N}=1 supersymmetric 
	  gauge theory. We first raise several reasons on why we 
	  are interested in supersymmetry. Then, we begin with 
	  some detail descriptions on supersymmetric gauge theory, 
	  including renormalization and anomalies. Later we focus 
	  on Seiberg Witten theory. 
   \end{abstract}
   
\chapter{Why Supersymmetry?}
	  This chapter is a brief illustration on why we need introduce 
	  supersymmetry to a theory. 
	
\chapter{The Description of N=1 Supersymmetry Gauge Theory}\cite{ryder1996quantum}

    \begin{align}
    	\int \dd^2{\tht}\frac{-i}{8\pi}\tau \text{tr} W_{\a}W^{\a}+ cc.
    \end{align}

\chapter{Renormalization}

\chapter{Anomalies}

\chapter{Seiberg Witten Theory}

\appendix

\chapter{Notation Convention}

  \section{4 dimension Gamma Matrice}
    \begin{equation}
       g^{\m\n} = \h^{\m\n} = \text{diag}(1,-1,-1,-1)
    \end{equation}
    
\printbibliography
\end{document}
