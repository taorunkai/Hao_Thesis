\documentclass{fduthesis-en}
\fdusetup{
	style/bibresource={test.bib},
	%style/automakecover=false,
	info={
		title={\textit{N} = 1 超对称规范理论综述},
		title*={Review of \textit{N} = 1 Supersymmetry Gauge Theory},
		author={张昊},
		supervisor={Satoshi Nawata},
		department={物理系},
		major={物理学},
		studentid = {14307110144},
		keywords* = {Supersymmetry Gauge Theory, Seiberg Witten Theory}
	}
}
\ExplSyntaxOn
\tl_set:Nn \c__fdu_def_name_thesis_type_tl{ 本科毕业论文 }
\ExplSyntaxOff

\begin{document}
\tableofcontents
\begin{abstract}
	This paper is a summary of \textit{N}=1 Supersymmetry 
	Gauge Theory. We first raise several reason on why we 
	are interested in Supersymmetry. Then, we begin with 
	some detail description on Supersymmetry Gauge Theory, 
	including Renormalization and Anomalies. Later we focus 
	on Seiberg Witten Theory. 
\end{abstract}
\chapter{Why Supersymmetry?}
	This chapter is a brief introduction on why we discuss 
	Supersymmetry. 
	
\chapter{The Description of N=1 Supersymmetry Gauge Theory}
\cite{ryder1996quantum}
\chapter{Renormalization}
\chapter{Anomalies}
\chapter{Seiberg Witten Theory}
\chapter{Appendix}
  \section{Notation Convention}
\printbibliography
\end{document}
