\documentclass[type = bachelor]{fduthesis-en}
\usepackage{tikz}
\usetikzlibrary{arrows,snakes,backgrounds}
\input{mathdef}
\fdusetup{
	style/bibresource={test.bib},
	%style/automakecover=false,
	info={
		title={\textit{N} = 1 超对称规范理论综述},
		title*={Review of \textit{N} = 1 Supersymmetric Gauge Theory},
		author={张昊},
		supervisor={Satoshi Nawata},
		department={物理系},
		major={物理学},
		studentid = {14307110144},
		keywords* = {Supersymmetric Gauge Theory, Seiberg Witten Theory}
	}
}


\begin{document}
  \tableofcontents
  \begin{abstract}
	  This paper is a summary of \textit{N}=1 supersymmetric 
	  gauge theory. We first raise several reasons on why we 
	  are interested in supersymmetry. Then, we begin with 
	  some detail descriptions on supersymmetric gauge theory, 
	  including renormalization and anomalies. Later we focus 
	  on Seiberg Witten theory. 
   \end{abstract}
   
\chapter{Why Supersymmetry?}
	  This chapter is a brief illustration on why we need introduce 
	  supersymmetry to a theory. 
	
\chapter{The Description of N=1 Supersymmetry Gauge Theory}

\noindent\textbf{\emph{Motivation \& problems}}

\noindent\textbf{Motivation \& problems}

Given an asymptotically-free gauge theory, we want to understand IR physics, which is difficult because the theory is strongly-coupled.\\

\begin{tikzpicture}[>=stealth']
\draw[->] (0,0) -- (4,0) node [right] {$\mu$};
\draw[->] (0,0) -- (0,2) node [above] {$g^2$};
\draw (1pt,1) -- (-1pt,1) node[left] {$1$};
\draw (1,1pt) -- (1,-1pt) node[below] {$\Lambda$} ;
\end{tikzpicture}

Eg.QCD

IR is quite different from UV.

UV: free theory of quarks and gluons

(deep)IR: free theory of pions and glueballs
\bigskip\\
\textbf{Strategy}

First,we make edugated guess for  long distance fields and symmetries; then write down all posssible effective actions(EA), which must be consistent with symmetries and selection rules. Finally, check original guess by using predictions from effctive actions. This strategy works very well in susy theories because of holomorphy and non-renormaliztion theorem.
\bigskip\\
\textbf{Introduction}

1.This lecture uses superspace technology

2.The use of these non-renormaliztion theorems to find exact (non-perturbative) LEEA will be one of the goal

3.This lecture will stick to 4d $N=1$ susy, but applies equally to many dimension and susy as well as SUGRA
\bigskip\\
\textbf{Outline}

1.Effective action

2.Symmetries, holomorphy,spurion analysis

3.Non-renormaliztion theorem for chiral superfield

4.Anomalies

5.Non-renormaliztion theorem for vector superfield

6.4d N=1 pure YM

7.4d N=1 SQCD

8.a-theorem
\section{Wilsonian Effective action}
$ S_{\mu} = \int d^{4}x L_{\mu}(\phi)$ describes physics at $E<\mu$, obtained by performing Feynman path integral over the short distance fluctuations of the fields on length scale $x<\frac{1}{\mu}$.

We make assumptions that it's local on scale $x \geq \frac{1}{\mu}$ and unitary for processes involving energies $E < \mu $.

For $E\sim\mu$, classical couplings(tree level) in $S_{\mu}$ describe effective couplings and masses (not renormalized by loops since short distance degree of freedom already integrated out).
\bigskip\\
Physical processes at $E \ll \mu$ will get quantum corrections due to fluctuations of modes in EA with energies between $E$ and $\mu$.
\bigskip\\
This corrections can be absorbed in coupling to define new EA at lower scale $E$.

RG: change in $S_{\mu}$ as $\mu$ decreases.

$S_{\mu} = \int d^{4}x \sum_{i} g_{i}(\mu)\mathcal{O}_{i}$, where $g_{i}$ is effective coupling and $\mathcal{O}_{i}$ is some operator.

RG equation is given by 
\begin{equation}
\label{RGeq}
\mu \frac{\partial g_{i}}{\partial \mu} = \beta_{i}(g_{F},\mu)
\end{equation}

\begin{tikzpicture}[>=stealth']
\draw [->] (-1,0) -- (4,0) node [right] {$g_1$};
\draw [->] (0,-1) -- (0,4) node [above] {$g_2$};
\end{tikzpicture}

Despite picture, RG flow is not reversible! There are infinitely many operators and couplings "integrated out" along each flow!

IR EAs are $S_{\mu}$ near an IR fixed point.($\mu$ is finite, but small compared to all other scales.)

One-particle irreducible EA is not equivalent to $S_{\mu=0}$
\bigskip\\
If IR fixed point is free,
\begin{equation}
\label{Sfree}
S_{free} = 
\end{equation}
These fields should scale to keep $S_{free}$ independent of scale $\mu_{0}$.

i.e. repalce scale

$$\mu_{0} \rightarrow \mu = (\frac{\mu}{\mu_{0}})\mu_{0}, dx_0 \rightarrow (\frac{\mu_0}{\mu})dx$$

$$E_0 \rightarrow (\frac{\mu}{\mu_0})E$$

$$\partial_0 \rightarrow (\frac{\mu}{\mu_0})\partial$$
\bigskip\\
rescaling

$$\phi_0 \rightarrow (\frac{\mu}{\mu_0})\phi$$

$$\psi_0 \rightarrow (\frac{\mu}{\mu_0})^{\frac{3}{2}}\psi$$

$$F^{\mu\nu}_0 \rightarrow (\frac{\mu}{\mu_0})^2F^{\mu\nu}$$

leaves $S_{free}$ as it was.
\bigskip\\
In general, if operator $\mathcal{O}_i$ scales as $\mathcal{O}_i \rightarrow (\frac{\mu}{\mu_0})^{\Delta_i}\mathcal{O}_i$, where $\Delta_i$ is called scaling dimension.

$\rightarrow$

\noindent So, $\Delta_i > 4$, $\mathcal{O}_i$ irrelevant, less important at IR;

$\Delta_i = 4$, $\mathcal{O}_i$ marginal;$\mu^{4-\Delta_i}\lambda_i(\mu)\mathcal{O}_i$\\
$$\mu\frac{d\tilde{\lambda}_i}{d\mu}=(4-\Delta_i)\tilde{\lambda}_i+\beta_i(\tilde{\lambda}_j)$$
where the first term on the right hand side is the classical scaling, while the second term is quantum correction from fluctuations $\mu +d\mu>E>\mu$ $\Rightarrow$ finite (no UV or IR divergence)
\\E.g. $L=(\partial\phi)^2+\lambda\phi^4 \Rightarrow \mathcal{O}=\phi^4, \Delta = 4$
\\$$\Rightarrow \lambda(\mu)=\frac{\lambda(\mu_0)}{1+ln(\frac{\mu_0}{\mu})\lambda(\mu_0)}$$
\\$\Rightarrow$ IR free theory\\
\bigskip Quantum corrections to kinetic terms(wave function renormalized)
\\e.g. write $$S_{\mu}=\int d^4x{Z(\mu)(\partial\phi)^2+\mu^2m^2(\mu)\phi^2+\lambda(\mu)\phi^4+...}$$\\
$Z$ is the renormalization of the wave function and can be absorbed in redefinition of fields $\phi \rightarrow \phi_{C.N.}\equiv \sqrt{Z}\phi$, where C.N. stands for canonically normalized.
$$\Rightarrow S_{\mu}=\int d^4x{(\partial\phi_{C.N.})^2+\mu^2m_{C.N.}^2(\mu)\phi_{C.N.}^2+\lambda_{C.N.}(\mu)\phi_{C.N.}^2+...}$$

with $m_{C.N.}(\mu)=\frac{m(\mu)}{\sqrt{Z(\mu)}}$, $\lambda_{C.N.}(\mu)=\frac{\lambda(\mu)}{Z^2(\mu)}$

Thus physical mass $m_{phys}=\mu m_{C.N.}(\mu)=\frac{\mu m(\mu)}{\sqrt{Z(\mu)}}$($\mu\rightarrow0$limit)

\bigskip As flow to IR,eventually reach a point where $m_{eff}=\mu m_{C.N.}(\mu)>\mu$

$\Rightarrow$ mass term dominates kinetic term and fixes $\phi$ to be constant.

So, for $\mu<m_{eff}, \phi\simeq const.$, and drop kinetic terms = "integrate out" $\phi$

massless fields, though, have degrees of freedom for any $\mu$.

\section{Types of IR fixed points}
There are three types of IR fixed points. The first one is the trivial one: all fields massive so for $\mu<$masses all integrated out and no propagating degree of freedom. The second is the free one:all massless fields are non-interacting in far IR.(e.g. $\lambda\phi^4$ theory, QED) The last is the interacting one: in 4d, we have no good description of interacting theories of massless particles(i.e. non-Lagrangian)

Coleman-Gross found that any theory of scalars, spinors and U(1) gauge fields is IR free. This gives a large class of IR-free theories(We will see later that only others are non-Abelian gauge theory with many massless flavors.)

Take field content of IR effective theory to be scalar $\phi^n$, complex Weyl spinor $\psi_{\alpha}^a$, U(1) vector $A_{\mu}^A$

${n,a,A}$ just to label different fields

\bigskip General form of leading (relevant) IR action

$S_{eff}=$

where $$D_{\mu}\chi=(\partial_{\mu}+q_AA_{\mu}^A)\chi$$

$$F_{\mu\nu}^A=\partial_{\mu}A_{\nu}^A-\partial_{\nu}A_{\mu}^A$$

$$\tilde{F}_{\mu\nu}^A=\frac{1}{2}\epsilon_{\mu\nu\rho\sigma}F^{A\rho\sigma}$$ "dual field strength"

Constraint $g_{mn}$ real symmetric positive definite

$g_{AB}$

V bounded below

It's convenient to define generalized $\mathbb{C}$ coupling

$$\tau_{AB}(\phi)=\frac{\theta_{AB}}{2\pi}+i\frac{4\pi}{g_{AB}^2}$$

The gauge kinetic terms written 

$\frac{i\tau}{16\pi^2}(\mathcal{F}^2)+c.c.$

where $\mathcal{F}^{\mu\nu}=\frac{1}{2}F^{\mu\nu}-\frac{i}{2}\tilde{F}^{\mu\nu}$ "self-dual"

\bigskip
-Meaning of $\tau_{AB}$

Since fields are free in IR, what is the meaning of gauge coupling $\tau$?

1) charged $\phi$ or $\psi$ are massless

1-loop running of U(1) coupling 

$\Rightarrow$ IR free

\bigskip
(picture)

2) all charged fields massive 

U(1) coupling stops running for $\mu<$lightest charged particle(since integrated out)

\bigskip
(picture)

In this case, $\tau$ measures strength of coupling only to massive (calssical) sources

\bigskip
-$\theta$-angles, sensitive to instanton number of field configs.

In presence of electric-magnetic massive charges, can have non-trivial $\theta$-effects.

\bigskip
-Electric-magnetic duality

field redefinition in U(1) theories

$$\tau\rightarrow\frac{A\tau+B}{C\tau+D}$$ 

\[
\left(
\begin{array}{cc}
A & B \\
C & D 
\end{array}
\right)
\] 

\section{Supersymmetric Effective Actions}
susy $\{Q_{\alpha},\bar{Q}_{\dot{\alpha}}\}=2\sigma^{\mu}_{\alpha\dot{\alpha}}P_{\mu}:=2P_{\alpha\dot{\alpha}}$

$\Rightarrow  \langle\psi|H|\psi\rangle = \sum_{\alpha}\left|Q_{\alpha}|\psi\rangle\right|^2+\sum_{\dot{\alpha}}\left|\bar{Q}_{\dot{\alpha}}|\psi\rangle\right|^2 \geq 0$

susy preserved $\Longleftrightarrow$ $V=0$ 

\begin{tikzpicture}[>=stealth']
\draw [->] (0,0) -- (4,0);
\draw [->] (0,0) -- (0,3);
\end{tikzpicture}

$V(\phi)=\left|\frac{dW}{d\phi}\right|^2$

\bigskip
Supermultiplets

(to be finished) Impose some conditions to get basic irreps.

Superspace: replace $Q_{\alpha}$, $\bar{Q}_{\dot{\alpha}}$ $\rightarrow$ diff op in supercoord
\begin{equation}
Q_{\alpha}=\frac{\partial}{\partial\theta^{\alpha}}-i\sigma^{\mu}_{\alpha\dot{\alpha}}\bar{\theta}^{\dot{\alpha}}\frac{\partial}{\partial x^{\mu}}
\end{equation}
\bigskip
Examples of irreps

\noindent1) chiral rep $[\bar{Q}_{\dot{\alpha}}, \mathcal{O}]=0$ or $\bar{D}_{\dot{\alpha}}\mathcal{O}=0$ in superspace
\begin{equation}
\Phi = \phi(y)+\sqrt{2}\mathcal{O}\psi(y)+\mathcal{O}^2F(y)
\end{equation}
2) Impose $\mathcal{O}:=V$ is real with $V \rightarrow V+i(\Lambda-\bar{\Lambda})$ symmetry with $\lambda = $ chiral superfield

$\Rightarrow$ gauge multiplet in superspace

bottom operator and impose it's real with

\begin{equation}
\label{FZmultiplet}
\bar{D}^{\dot{\alpha}}T_{\alpha\dot{\alpha}}=\bar{D}^{\alpha}T_{\alpha\dot{\alpha}}=0
\end{equation}

\begin{equation}
T_{\mu}=j^{\mu}_R-i\theta(S_{\mu}+...)+i\bar{\theta}(\bar{S}_{\mu}+...)+\theta\sigma^{\nu}\bar{\theta}(2T_{\mu\nu}+...)
\end{equation}
More generally, Ferarra-Zumino multiplet has $\bar{D}^{\dot{\alpha}}T_{\alpha\dot{\alpha}}=D_{\alpha}X$ $X$: chiral superfield

\begin{equation}
F_x=\frac{2}{3}T^{\mu}_{ \ \mu}+i\partial_{\mu}j^{\mu}_R
\end{equation}
violation of scale and R-symmetry tied together. "multiplet of anomalies"

\bigskip
\noindent-Scalars, Spinors and Vector fields appear in chiral superfield as
\begin{equation}
\Phi(x,\theta)=\phi(x)+\psi(x)\theta+F(x)\theta\theta
\end{equation}
They appear in vector superfield as 
\begin{equation}
V(x, \theta, \bar{\theta})=\bar{\theta}\sigma^{\mu}\theta A_{\mu}(x)+\bar{\theta}^2\theta\lambda(x)+...
\end{equation}
chiral field-strength 
\begin{equation}
W_{\alpha}(x,\theta)=\lambda_{\alpha}(x)+(\sigma^{\mu\nu}\theta)\mathcal{F}_{\mu\nu}(x)+...
\end{equation}
\begin{equation}
W_{\alpha}~\bar{D}^2e^{-V}D_{\alpha}e^V
\end{equation}
\begin{equation}
D_{\alpha}W^{\alpha}=\bar{D}^{\dot{\alpha}}W_{\alpha}=0
\end{equation}
gauge inv.
\begin{equation}
e^{-V} \rightarrow e^{-i\bar{\Lambda}}e^{-V}e^{i\Lambda}
\end{equation}
\begin{equation}
W_{\alpha} \rightarrow e^{-i\Lambda}W_{\alpha}e^{i\Lambda}
\end{equation}
where $\Lambda$ is some chiral superfield.

\bigskip
-Supersymmetric actions given by 
\begin{equation}
L = \int d^2\theta d^2\bar{\theta}\widetilde{K}+\int d^2\theta \widetilde{W}+h.c.
\end{equation}
where $\widetilde{K}$ is general superfield, dim$\widetilde{K}=2$; $\widetilde{W}$ is chiral superfield, dim$\widetilde{W}=3$.
\begin{equation}
L = \widetilde{K}_{ij}\partial_{\mu}\phi^{i}\partial^{\mu}\bar{\phi}^j+...-W_{ij}\psi^i_{\alpha}\psi^j_{\beta}\epsilon^{\alpha\beta}-K^{ij}\partial_iW\partial_j\bar{W}
\end{equation}
-Kinetic terms
\begin{equation}
\int d^2\theta d^2\bar{\theta} \bar{\Phi}\Phi \longrightarrow \int d^2\theta d^2\bar{\theta} \bar{\Phi}e^V\Phi, \int d^2\theta Tr(W_{\alpha}W^{\alpha})
\end{equation}
-General superfield
\begin{equation}
\widetilde{K}=K(\bar{\Phi}e^V, \Phi)+\widetilde{K}(\bar{\Phi}e^V, \Phi, D, \bar{D}, \partial_{\mu})
\end{equation}
where $K$ is Kahler potential and $D$, $\bar{D}$ and $\partial_{\mu}$ are arbitrary derivatives

\noindent-Chiral superfield
\begin{equation}
\widetilde{W}=W(\Phi)+\tau(\Phi)Tr(W_{\alpha}W^{\alpha})+\theta(W^4)+W^{\prime}(\Phi, \partial_{\mu})+...
\end{equation}
-To see which terms, we need to keep in IR action

$$x_0\rightarrow(\frac{\mu_0}{\mu})x, \theta_0\rightarrow(\frac{\mu_0}{\mu})\theta$$

$$\partial^0_{\mu}\rightarrow(\frac{\mu}{\mu_0})\partial_{\mu}, \partial_{\theta_0}\rightarrow(\frac{\mu}{\mu_0})\partial_{\theta}$$

$$dx_0\rightarrow(\frac{\mu_0}{\mu})dx, d\theta_0\rightarrow(\frac{\mu_0}{\mu})d\theta$$

superfield scales in the same way as lowest comp.

$$\Phi_0\rightarrow(\frac{\mu}{\mu_0})\Phi$$

$$W^{\alpha}_0\rightarrow(\frac{\mu}{\mu_0})^{\frac{3}{2}}W^{\alpha}$$

\noindent-Chiral superfield 
$$\mathcal{O}_{i0}\rightarrow(\frac{\mu}{\mu_0})^{\Delta_i}\mathcal{O}_i$$
$$\Rightarrow \int d^4x\int d^2\theta_0\mathcal{O}_{i0}\rightarrow(\frac{\mu}{\mu_0})^{\Delta_i-3}\int d^4x d^2\theta \mathcal{O}_i$$
$\Delta_i=3$ is marginal chiral superfield

\noindent-General superfield 
$$\eta_{i0}\rightarrow(\frac{\mu}{\mu_0})^{\Delta_i}\eta_i$$
$$\Rightarrow \int d^4x_0 d^2\theta_0 d^2\bar{\theta_0}\eta_{i0}\rightarrow(\frac{\mu}{\mu_0})^{\Delta_i-2}\int d^4x d^2\theta d^2\bar{\theta}\eta_i$$
$\Delta_i=2$ is marginal superfield.

\noindent-Write down relevant term
\begin{equation}
S_{\mu}=\int d^4x d^2\theta Z(\mu) \bar{\Phi}e^V\Phi+(\int d^2\theta\dot{\tau}(\mu)TrW^2+W(\Phi)+h.c.)
\end{equation}
Remark

For U(1) gauge group, we can add Faget-Iliopoulos term $\int d^4x \xi \int d\theta_{\alpha}W^{\alpha}$.

\noindent Example
$$L=\int d^4\theta K + \int d^2\theta W + h.c.$$ for chiral superfield $X$, where $K=\bar{X}X$, $W=hX^n$, dim$X=1$

In 4d: $h$ is relevant, $n\leq2$; $h$ is irrelevant, $n\geq4$; $h$ is marginal irrelevant,$n=3$

We can perturbative superpotential $W=\sum_{p=0}^nh_pX^p$

$V=\left|\frac{dW}{dX}\right|^2$ has $(n-1)$ susy vacua (spout susy breaking for $n=1$)

\section{Symmetrie and Holomorphy}
-Continuous symmetry with parameter ($\epsilon \in \mathbb{R}$) and generator $Q$
\begin{equation}
\mathcal{O}(0)\rightarrow\mathcal{O}(\epsilon)=e^{i\epsilon Q}\mathcal{O}(0)e^{-i\epsilon Q}\simeq \mathcal{O}(0)+i\epsilon[Q,\mathcal{O}(0)]+...
\end{equation}
-Main Symmetries: Poincare

-space-time translation generated by $P_{\mu}$ ($x^\mu\rightarrow x^\mu+\epsilon^\mu$)
\begin{equation}
\phi(x)\rightarrow\phi(x-\epsilon)\Rightarrow \delta\phi=-\epsilon^\mu\partial_\mu\phi\Rightarrow[P_\mu,\phi(x)]=i\partial_\mu\phi(x)
\end{equation}

In local field theory, $P_\mu=\int d^3x T_{\mu0}$, when $\partial^\mu T_{\mu\nu}=0$ and $T_{\mu\nu}=T_{\nu\mu}$, where $T_{\mu\nu}$ is the conserved EM tensor.

\begin{equation}
T^{\mu\nu}(x)=2\frac{\delta}{\delta g^{\mu\nu}}[\int d^4xL]_{g_{\mu\nu}=\eta_{\mu\nu}}
\end{equation}

-Lorentz transformation generated by $M_{\mu\nu}$

\bigskip
-Poincare + locality
$\Rightarrow S=\int d^4x L(\phi(x))$ with $\phi(x)$ transforming as finite-dimension representation of Lorentz group. $L$ is a Lorentz scalar.

\bigskip
\noindent-Other spacetime symmetries

-scale invariance ($x^\mu \rightarrow e^\epsilon x^\mu$)

We used this in our RG discussion with $e^\epsilon =\frac{\mu}{\mu_0}$, and saw that operator of dimension $\Delta$ obeys transformation 
\begin{equation}
\phi(x) \rightarrow \tilde{\phi}(e^\epsilon x)=e^{-\epsilon \Delta}\phi(x)
\end{equation}
or
\begin{equation}
\tilde{\phi}(x)=e^{-\epsilon \Delta}\phi(e^{-\epsilon}x)=\phi(x)-\epsilon(\Delta\phi(x)+x\partial\phi)+\mathcal{O}(\epsilon^2)
\end{equation}
\begin{equation}
\delta\phi=-\Delta\phi-x\partial\phi
\end{equation}
If we say it is generated by the dilatation operator $D$, then
\begin{equation}
[D, \phi(x)]=i\Delta\phi(x)+ix^\mu\partial_\mu\phi(x)
\end{equation}
Using Noether theorem, we can show that
\begin{equation}
D_\mu = T_{\mu\nu}x^\nu \Rightarrow \partial^\mu D_\mu = T^{\mu}_{\ \mu}
\end{equation}
Thus scaling is a symmetry, if and only if $T^{\mu}_{\ \mu}=0$

\noindent-Supersymmetry
\\Fermionic charges $Q_{\alpha}=\int d^3x S^0_\alpha$ with $\partial_\mu S_\alpha^\mu=0$, where $S_\alpha^\mu$ are supercurrents.

-By Coleman-Mandula, Haag-Lopusza\'{n}ski-Sohnius, all other symmetries of (local, d$>2$,interacting) quantum field theories are internal symmetries. Scalar conserved charges $Q_i$ satisfy Lie algebra
\begin{equation}
[Q_i,Q_j]=if^k_{ij}Q_k
\end{equation}
-Simplest U(1) symmetries generated by commuting charges
\begin{equation}
[Q_i,Q_j]=0, \forall i, j
\end{equation}
\bigskip
Examples

\noindent1. Free $\mathbb{C}$ scalar 
\begin{equation}
L=\partial_{\mu}\bar{\phi}\partial^\mu\phi
\end{equation}
with symmetry 
\begin{equation}
\phi(x)\rightarrow e^{i\epsilon q}\phi(x)
\end{equation}
We can get
\begin{equation}
[Q,\phi(x)]=q\phi(x)
\end{equation}
where
\begin{equation}
Q=\int d^3x J^0 
\end{equation}
with 
\begin{equation}
\partial^\mu J_\mu = 0
\end{equation}
\begin{equation}
J_\mu=i(\bar{\phi}\partial_\mu\phi-\phi\partial_\mu\bar{\phi})
\end{equation}
2. Free spinor
\begin{equation}
L=\bar{\psi}\slashed{\partial}\psi, \psi_\alpha \rightarrow e^{i\epsilon q}\psi_\alpha
\end{equation}
$$\Rightarrow J_\mu = \bar{\psi}\sigma_\mu\psi$$

Note: if $Q_1$ and $Q_2$ are two U(1) symmetries, then their real linear combination 
\begin{equation}
Q=a_1Q_1+a_2Q_2
\end{equation}
is also a symmetry, so if there are many U(1) symmetries, we have to pick the basis and renormalization.

\bigskip
Non-Abelian symmetries $f^k_{ij}\neq0$

Once basis and renormalization of $Q_i$'s are fixed, $f^k_{ij}$ is specified. For example, flavor symmetries of massless QCD
\begin{equation}
L=\sum_{i=1}^{N_f}(\bar{\psi}^i\slashed{D}\psi_i+\bar{\tilde{\psi}}_i\slashed{D}\tilde{\psi}^i)+Tr(F_{\mu\nu}F^{\mu\nu})
\end{equation}
with $\psi_i$ transformation in representation R of gauge group G ,"quark";
$\tilde{\psi}_i$ transformation in representation $\bar{R}$ of gauge group G, "anti-quark"

Then $U(N_f)_L=U(1)\times SU(N_f)_L$
$$\psi_i \rightarrow g_i^{\ j}\psi_j$$
$$\tilde{\psi}^i\rightarrow\tilde{\psi}^i$$
where $g_i^{\ j}$ is a $N_f\times N_f$ unitary matrix.

$U(N_f)_R=U(1)^\prime\times SU(N_f)_R$
$$\psi_i \rightarrow \psi_i$$
$$\tilde{\psi}^i\rightarrow g^i_{\ j}\tilde{\psi}^j$$
Note: L = "left", R = "right" and combination $\Psi_i=$

If $Q^a_L$ and $Q^a_R$ are generators of $SU(N_f)_L\times SU(N_f)_R$, $Q$ and $Q^\prime$ are generators of $U(1)\times U(1)^\prime$, then we usually define combinations
\begin{equation}
Q_A=Q-Q^\prime
\end{equation}
\begin{equation}
Q_B=Q+Q^\prime
\end{equation}
We call them "axial" $U(1)_A$ and "baryon number" $U(1)_B$ respectively.
\begin{equation}
U(1)_A: \Psi_i \rightarrow e^{i\epsilon\gamma_5}\Psi_i
\end{equation}
\begin{equation}
U(1)_B: \Psi_i \rightarrow e^{i\epsilon}\Psi_i
\end{equation}
Note: we can also define combinations
\begin{equation}
Q_v^a=Q_L^a+Q_R^a, Q_A^a=Q_L^a-Q_R^a
\end{equation}
$Q_v^a$ generate "diagonal subgroup" $SU(N_f)_v\subset SU(N_f)_L\times SU(N_f)_R$

Algebra of $Q_L$ and $Q_R$
\begin{align}
[Q_L^a,Q_L^b]&=if^{ab}_cQ^c_L\\
[Q_R^a,Q_R^b]&=if^{ab}_cQ^c_R\\
[Q_L^a,Q_R^b]&=0
\end{align}
Also, we can get the algebra of $Q_v$ and $Q_A$
\begin{align}
[Q_v^a,Q_v^b]&=if^{ab}_cQ_v^c\\
[Q_A^a,Q_A^b]&=if^{ab}_cQ_v^c\\
[Q_A^a,Q_v^b]&=if^{ab}_cQ_A^c
\end{align}
-Mass terms in QCD

break flavor symmetries:

To be gauge invariant, we need to match $R$ and $\bar{R}$ representations.
\begin{equation}
L_{mass}=m_i^{\ j}\tilde{\psi}^i\psi_j+c.c.
\end{equation}
With this term, the separate left and right flavor symmetries are not conserved
\begin{equation}
U(N_f)_L\times U(N_f)_R\rightarrow U(N_f)_v
\end{equation}
$SU(N_c)$ non-susy QCD phases: $f_n$ of $N_f$

(picture here)
\section{Holomorphy in Supersymmetry}
-IR effective action of 4d $N=1$
\begin{align}
\notag
L_{eff}&=\int d^2\theta [W_{eff}(\tilde{\Phi})+\tau_{eff}(\tilde{\Phi})Tr\widetilde{W}_\alpha\widetilde{W}^\alpha]+c.c.\\
&+\int d^2\theta d^2\bar{\theta}Z_{wf}\bar{\tilde{\Phi}}e^V\tilde{\Phi}
\end{align}
-Suppose UV theory is A.F. susy gauge theory
\begin{align}
\notag
L_{UV}&=\int d^2\theta [W_{UV}(\Phi)+\tau Tr(W^2)]+c.c.\\
&+\int d^2\theta d^2\bar{\theta}\bar{\Phi}e^V\Phi
\end{align}
with
\begin{equation}
W_{UV}(\Phi)=\lambda_2\Phi^2+\lambda_3\Phi^3
\end{equation}
-Here $\{\Phi,W_\alpha\}$ in UV is not necessarily the same as $\{\tilde{\Phi},\tilde{W}_\alpha\}$ in IR, $\{\lambda_i,\tau\}$ are $\mathbb{C}$ parameters\\
The question: How do $W_{eff}$, $T_{eff}$ and $Z_{wf}$ depend on the parameters $\{\lambda_i,\tau\}$ of the theory?\\
-Spurion analysis:
\\One can consider $\{\lambda_i,\tau\}$ as chiral superfields and treat
$$\lambda_i\rightarrow\lambda_i(y,\theta), \tau \rightarrow \tau(y, \theta)$$
where
$$y=x+i\bar{\theta}\sigma\theta$$ 
The parameter can be understood as vacuum expectation value (VEV)
\begin{align}
\notag
\lambda_i&=\langle\lambda(y,\theta)\rangle
\\T&=\langle T(y,\theta)\rangle
\end{align}
Since chiral superfields can only enter holomorphically in $W_{eff}$ and $\tau_{eff}$, $\{\lambda_i,T\}$ can only enter holomorphically as well.\\
Let's turn off vector multiplet. Since this is just a theory of scalars, spinors which are IR free for weak enough coupling. We assume
\begin{equation}
\{\tilde{\Phi}\}=\{\Phi\}
\end{equation}
i.e. IR field content is the same as UV field content.\\
Now we can see the "power of holomorphy": Say
\begin{equation}
W_0=\lambda \Phi
\end{equation}
Then, we can say that there is a U(1) global symmetry $\Phi\rightarrow e^{i\epsilon}\Phi$, broken by $\lambda$. Thus, thinking of $\lambda$ as a superfield, we can restore this symmetry.\\
Without holomorphy, we could write infinitely many possible contributions to effective superpotential
\begin{equation}
W_{eff}=...+\frac{\bar{\lambda}^2}{\Phi^2}+\frac{\bar{\lambda}}{\Phi}+\lambda\Phi+\lambda^2\bar{\lambda}\Phi+...+\lambda e^{-\frac{1}{|\lambda|^2}}\Phi+...
\end{equation}
But with holomorphy, the only terms allowed are of the form
\begin{equation}
W_{eff}=\sum_\alpha(\lambda\Phi)^\alpha, \alpha>0
\end{equation}
This already rules out non-perturbative contributions. Also, $\alpha>0$ since the theory is free as $\lambda\rightarrow0$. Note that K\"{a}hler potential is not holomorphic, so don't get these restrictions.

Seiberg summarized this argument prescriptively, constrain the effective superpotential by holomorphy in UV couplings, global symmetries broken by the coupling and smoothness of the physics in weak coupling limits.
\section{Non-renormalization of the superpotential}
Start with Wess-Zumino model of one chiral superfield
\begin{equation}
W_{\mu0}=\frac{1}{2}\lambda_2\Phi^2+\frac{1}{3}\lambda_3\Phi^3
\end{equation}
UV K\"{a}hler term is invariant under $U(1)\times U(1)_R$, we can assign charge
\begin{align}
\notag U(1)\times U(1)_R\\
\notag \Phi 1 1\\
\theta 0 +1\\
d\theta 0 -1\\
W 0 2\\
\lambda_2 -2 0\\
\lambda_3 -3 -1
\end{align}
Holomorphy and spurion analysis tell us 
\begin{equation}
W_\mu=W(\Phi, \lambda_2, \lambda_3)
\end{equation}
Symmetry dictates
\begin{equation}
W_\mu=\lambda_2\Phi^2g(\frac{\lambda_3\Phi}{\lambda_2})
\end{equation}
$\lambda_3\rightarrow0$ limit,
\begin{equation}
W_\mu=\sum_{n\geq0}g_n\lambda_2^{1-n}\lambda_3^n\Phi^{n+2}
\end{equation}
$\lambda_2\rightarrow0$ limit,
\begin{equation}
W_\mu=g_0\lambda_2\Phi^2+g_1\lambda_3\Phi^3
\end{equation}
At $\lambda_3=0$, theory is free (massive), mass runs canonically. In our convention, mass runs as $(\frac{\mu_0}{\mu})$ in free theory, so we get
\begin{equation}
g_0\lambda_2=\frac{1}{2}(\frac{\mu_0}{\mu})\lambda_2
\end{equation}
Thus,
\begin{equation}
g_0=\frac{1}{2}(\frac{\mu_0}{\mu}) 
\end{equation}
It's weak coupling.\\
Take $\lambda_3$ small, use perturbation theory to match $W_\mu\leftrightarrow W_{\mu0}$. Since $\Phi^3$ vertex appears in both proportional to the same coupling $\lambda_3$, they must match at tree level (classically), which gives $g_1=\frac{1}{3}(\frac{\mu_0}{\mu})^0$
\begin{equation}
W_\mu=\frac{1}{2}(\frac{\mu_0}{\mu})\lambda_2\Phi^2+\frac{1}{3}\lambda_3\Phi^3
\end{equation}
i.e. Couplings are not renormalized, no quantum corrections enter. (only classical scaling)\\
This non-renormalization result shows nocontradiction with our original assumption that the degree of freedom of IR is the same as that of UV. For $W_{\mu0}$ smal coupling, Coleman-Gross tells $W_\mu0$ IR-free. So $W_\mu=W_{\mu0}$ for all coupling. Wess-Zumino model has no non-trivial fixed points in RG flow at strong coupling.\\
Remark:\\
This argument does \emph{not} preclude the quantum generation of higher-derivative generalized superpotential terms.
\section{K\"{a}hler term renormalization}
Since K\"{a}hler term is not protected by this argument, it will get wave function renormalization (+ higher derivatives). So in order to see how the canonical couplings are renormalized, we define canonical chiral superfield by
\begin{equation}
\Phi_n\rightarrow\Phi_n^{C.N.}=\sqrt{Z_n(\mu)}\Phi_n
\end{equation}
Then if superpotential is
\begin{equation}
W_\mu=(\frac{\mu_0}{\mu})^{3-\Delta_v}\lambda_v\mathcal{O}_v
\end{equation}
with
\begin{equation}
\mathcal{O}_v=\Pi_n\Phi_n^{v_n}, \Delta_v=\sum_nv_n
\end{equation}
Then
\begin{equation}
W_\mu=(\frac{\mu_0}{\mu})^{3-\Delta_v}(\Pi Z_n^{-\frac{v_n}{2}})\lambda_v\mathcal{O}_v
\end{equation}
\begin{equation}
\lambda_v^{C.N.}(\mu)=(\frac{\mu_0}{\mu})^{3-\Delta_v}(\Pi Z_n^{-\frac{v_n}{2}})
\end{equation}
\begin{equation}
\mu\frac{d\lambda^{C.N.}}{d\mu}=\lambda_v^{C.N.}(\mu)(\Delta_v-3-\frac{1}{2}\sum_nv_n-\gamma_n(\mu))
\end{equation}
It's an exact RG equation, but we don't know how to compute $\gamma_n(\mu)$ exactly.
We define the anormalous dimension of $\Phi_n$ as 
\begin{equation}
\gamma_n(\mu)=\frac{d\ln Z_n}{d\ln\mu}
\end{equation}
Example

Theory with two chiral superfields $\Phi_1$, $\Phi_2$ and $W_{\mu0}=\lambda\Phi_1\Phi_2^2$, to get susy vacua
\begin{equation}
\frac{\partial W}{\partial\Phi_1}=\Phi_2=0, \frac{\partial W}{\partial\Phi_2}=\Phi_1\Phi_2=0
\end{equation}
And we can get $\Phi_2=0$ and $\Phi_1$ is arbitrary.

Whole moduli space of vacua: degenerate into energy, but inequivalent classical ground state. By non-renormalization theorem of superpotential, this conclusion does not change once quantum effects are taken into account. But quantum effect can renormalize K\"{a}hler potential and thus change metric on $\mathcal{M}$ from classical value.\\
Since
\begin{equation}
K_{\mu0}=\bar{\Phi}_1\Phi_1+\bar{\Phi}_2\Phi_2
\end{equation}
induce metric on $\mathcal{M}$
\begin{equation}
d^2s=d\Phi_1d\bar{\Phi}_1
\end{equation}
Spectrum at any $\langle\Phi_i\rangle$ is: massless $\Phi_1$, massive $\Phi_2$ with mass $\sim\left|\lambda\langle\Phi_1\rangle\right|$. At scale $>$ mass $\Phi_2$, 1-loop diagram of $\Phi_2$ contribute to $\Phi_1$ propagator.
\begin{equation}
K=\bar{\Phi_1}\Phi_1-(\#)\bar{\Phi}_1\Phi_1\left|\lambda\right|^2\ln\left|\frac{\Phi}{\mu_0}\right|^2+...
\end{equation}
For IR effective action scale $\mu<$ mass $\Phi_2$

As $\Phi_1\rightarrow0$, $K\rightarrow\infty$, $\Rightarrow$ canonically coupling $\rightarrow0$, $\Rightarrow$ 1-loop approximation becomes better near original of $\mathcal{M}$.\\
K\"{a}hler metric 
\begin{align}
\notag ds^2&=(\partial_i\partial_jK)d\Phi_id\bar{\Phi}_j\\
&=(const.-\left|\lambda\right|^2\log\Phi_1\bar{\Phi}_1)d\Phi_1d\bar{\Phi}_1
\end{align} 
It's singular at $\Phi_1=0$.
\\(picture)
\\What does () singularity mean?
\\The assumption under which we computed IR EA breaks down. We assume $\mu<$ mass $\Phi_2$, as $\left|\Phi_1\right|\rightarrow0$, this does not remain true for finite $\mu$.

General lesson: singularity in moduli space is a sign of new massless degree of freedom. To get a smooth description of physics, we need to include them.
\subsection{Gauge theory and moduli space of (classical) vacua}
\begin{equation}
L=\sum_f\int d^4\theta Q_f^\dagger e^{T_fV}Q_f+\int d^2\theta\tau TrW_\alpha W^\alpha+h.c.
\end{equation}
\begin{equation}
D^\alpha\sum_fQ_f^\dagger T_f^aQ_f-\frac{1}{g^2}\sum_a(D^a)^2
\end{equation}
\begin{equation}
D^a=g^2\sum_fQ_f^\dagger T_f^aQ_f
\end{equation}
\begin{equation}
V_D=\frac{1}{2g^2}\sum_a(D^a)^2
\end{equation}
susy vacua $V_D=0\Leftrightarrow D^a=0$, $a=1,\cdots,\left|G\right|$
Thus
\begin{align}
\mathcal{M}_{cl}&=\{\langle Q_f\rangle|D^a=0\}/\mathrm{gauge\ transformation}\\
&=\{\langle\mathrm{gauge\ invariant\ chiral\ superfileds}\rangle\}/\mathrm{classical\ ()}
\end{align}
Example\\
$U(1)$ with $Q$ charge $+1$\\
$U(1)$ with $\widetilde{Q}$ charge $-1$\\
SQED with $\mathcal{N}_f=1$ massless electron\\
\begin{equation}
V_D=g^2(\left|Q\right|^2-\left|\widetilde{Q}\right|^2)^2\geq0
\end{equation}
$\mathcal{M}_{cl}$ has
\[
\left|Q\right|=\left|\widetilde{Q}\right|
\]
$U(1)$ gauge transformation
\begin{equation}
Q\rightarrow e^{i\phi}Q, \widetilde{Q}\rightarrow e^{-i\phi}\widetilde{Q}
\end{equation}
() out by this transformation
\begin{equation}
dim_{\mathbb{R}}\mathcal{M}_{cl}=2\times(1+1)-1-1=2
\end{equation}
\begin{align}
\mathcal{M}_{cl}&=\{\langle Q\rangle\langle\widetilde{Q}\rangle|V_D=0\}/\mathrm{gauge\ transformation}\\
&=\{\langle M\rangle\}
\end{align}
where $M=Q\widetilde{Q}$ is the gauge invariant "meson".\\
(picture)\\
Higgs mechanism: two chiral superfields $Q$ and $\widetilde{Q}$. One eaten by broken $U(1)$ is the light field $\Leftrightarrow M$\\
SQED $U(1)$ with $\mathcal{N}_f$ flavor\\
$Q_f$ charge $+1$\\
$\widetilde{Q}_f$ charge $-1$
\begin{align}
\notag
\mathcal{M}_{cl}&=\{\langle Q_f\rangle\langle\widetilde{Q}_f\rangle|V_D=\Sigma_f\left|Q_f\right|^2-\left|\widetilde{Q}_f\right|^2=0\}\\
\notag&/mathrm{gauge\ transformation}\\
&=\{\langle M_{f\tilde{g}}=Q_f\widetilde{Q}_{\tilde{g}}\rangle\}
\end{align}
(picture)\\
In order to deal with the singularity at $M_{f\tilde{g}}=0$, we have two options:\\ 
(a) It's removed by quantum effects; \\
(b) It's associated with additional degree of freedom, becoming massless.\\
Classically we know it's option (b), photon and additional matte field become massless at $\langle Q_f\rangle=\langle\widetilde{Q}_f\rangle=0$, $U(1)$ unHiggsed there.\\
Since SQED is IR free, this also holds in quantum theory. We will see in SQCD other mechanisms, depending on $\mathcal{N}_f$ and $\mathcal{N}_c$.
\section{Anomalies}
In computing quantum corrections, we need to regulate in UV. Some symmetries of classical theory are always broken by the regularization. It can happen that even in the limit when the regulator is removed, the classical symmetry remains broken. This is called anomaly. Furthermore, we'll show that a class of these anomalies, the chiral anomalies are not just UV artifacts, but persist and have effects at all scales.
\subsection{Trace Anomalies}
Most important anomaly is scale invariance anomaly. Recall dilatation current $D^\mu$
\begin{equation}
\partial^\mu D_\mu=T^\mu_{\ \mu}
\end{equation}
Even when $T^\mu_{\ \mu}$ vanishes classically, it is a composite operator, so it must be regulated quantum mechanically.

\noindent Example: QCD with dimensional regularization
\begin{equation}
T^\mu_{\ \mu}=\frac{d-4}{4}Tr(F\wedge*F)+(1-d)i\bar{\psi}\slashed{D}\psi
\end{equation}
Classically, from $d=4$ and the equation of motion of $\psi$, we can get $T^\mu_{\ \mu}=0$\\
Quantum mechanically, we compute the background field\\
(picture)\\
\begin{equation}
T^\mu_{\ \mu}=\frac{b}{32\pi^2}Tr(F\wedge*F)
\end{equation} 
On the other hand, under scaling $x\rightarrow e^\epsilon x$
\begin{equation}
T^\mu_{\ \mu}=(\frac{\partial L}{\partial\epsilon}-4L)|_{\epsilon=0}=\frac{\partial(-\frac{1}{4g^2})}{\partial\epsilon}Tr(F\wedge*F)
\end{equation}
\begin{equation}
\frac{\partial g^{-2}}{\partial\epsilon}=-\frac{b}{8\pi^2}
\end{equation}
Under this scaling, $\mu_0\rightarrow e^{-\epsilon}\mu_0:=\mu$, $\epsilon=\ln(\frac{\mu_0}{\mu})$
\begin{equation}
\frac{1}{g^2(\epsilon)}-\frac{\Lambda}{g^2(0)}=-\frac{b\epsilon}{8\pi^2}=\frac{b}{8\pi^2}\ln(\frac{\mu}{\mu_0})
\end{equation}
\begin{align}
b<0&\Rightarrow \mathrm{IR\ free}\\
b>0&\Rightarrow \mathrm{UV\ free}
\end{align}
We can introduce strong-coupling scale
\begin{equation}
\left|\Lambda\right|:=\mu e^{-\frac{8\pi^2}{bg^2(\mu)}}
\end{equation}
$\mu$ independent at 1-loop\\
Computation of $b$ coefficient gives 
\begin{equation}
b=\frac{11}{6}T(adj)-\frac{1}{3}\sum_jT(R_j)-\frac{1}{6}\sum_aT(R_c)
\end{equation}
$i$: Weyl fermions in representation $R_i$\\
$a$: $\mathbb{C}$ scalars in representation $R_a$
\begin{equation}
T(R)=\frac{C(R)}{C(fund)}
\end{equation}
It's called the index of representation.
\begin{equation}
C(R)\delta^{ab}:=Tr_R(T^aT^b)
\end{equation} 
For classical groups\\
(table)\\
Thus, for $SU(\mathcal{N}_c)$ QCD with $\mathcal{N}_f$ flavors in fund\\ 
$2 \mathcal{N}_f$ Weyl fermions and zero bosons\\
$T(adj)=2\mathcal{N}_c$, $T(fund)=1$
\begin{equation}
b=\frac{11}{6}2\mathcal{N}_c-\frac{2}{3}\mathcal{N}_f=\frac{11\mathcal{N}_c-2\mathcal{N}_f}{3}
\end{equation}
It works for U(1) factors as well:\\
representations are all one dimensional with charge $q$
$T(U(1)adj)=0$, $T(R_q)=q^2$\\
$\Rightarrow$ All $U(1)$'s are IR-free.\\
One other gauge coupling: theta angle $\sim\frac{\theta}{16\pi^2}Tr(F\wedge F)$\\
Instanton number is an integer
\begin{equation}
\frac{1}{16\pi^2}\int Tr(F\wedge F)\in\mathbb{Z}
\end{equation}
\begin{equation}
\Rightarrow\theta\sim\theta+2\pi
\end{equation}
$\theta$ does not run perturbatively.\\
Recall
\begin{equation}
T(\mu)=\frac{\theta}{2\pi}+i\frac{4\pi}{g^2(\mu)}
\end{equation}
So we can define $\mathbb{C}$ RG invariant scale
\begin{equation}
\Lambda=\left|\Lambda\right|e^{i\frac{\theta}{b}}=\mu e^{2\pi i\frac{\tau(\mu)}{b}}
\end{equation}
\begin{equation}
\tau(\mu)=\frac{b}{2\pi i}\ln(\frac{\Lambda}{\mu})
\end{equation}
Power of holomorphy
\begin{equation}
2\pi i\mu\frac{d\tau}{d\mu}=f(\tau)\leftarrow\mathrm{holomorphic}
\end{equation}
However, $\theta$ does not run
\begin{equation}
\Rightarrow f(\tau) = const. \Rightarrow 2\pi i\mu\frac{d\tau}{d\mu}=16\pi^2g^{-3}\beta_g\ \mathrm{(1-loop exact)}
\end{equation}
\subsection{Chiral Anomalies}
Chiral symmetries are those in which the left-handed fermions ($\psi_\alpha$) transform differently than the right-handed fermions ($\bar{\psi}_{\dot{\alpha}}$). Since (in 4-dimension) they are complex conjugate, if $\psi_\alpha$ transforms in representation $R$, then $\bar{\psi}_{\dot{\alpha}}$ transforms in representation $\bar{R}$.

We want to compute quantum correction to 
\begin{equation}
\partial_\mu J^\mu_a=0\ \ \ \ \ \ \mathrm{with}\ J_a^\mu=\bar{\psi}\sigma^\mu T_a\psi 
\end{equation}
In 4-dimension, 1-loop diagram\\
(picture)\\
\begin{equation}
\partial_\mu J^\mu_a=\#\sum_iTr_{R_i}(T_a\{T_b,T_c\})\epsilon_{\mu\nu\rho\sigma}F_b^{\mu\nu}F_c^{\rho\sigma}
\end{equation}
Note: only need to trace massless fermions in loop since massive ones don't contribute.\\
Higher loops do not contribute.\\
If $\sum_vTr(T_a\{T_b,T_c\})=0$, no chiral anomalies. Only simple groups with $Tr(T_a\{T_b,T_c\})\neq0$ are $SU(\mathcal{N})_{\mathcal{N}\geq3}$. But when also $U(1)$ factors, it get non-zero.

In the presence of gauge and flavor symmetry, there are anomalies of three types: GGG, FGG and FFF, where G stands for gauge group and F stands for flavor group.

GGG\\
total anomaly must vanish or theory is sick.

FGG\\
(picture)
\begin{align}
\notag&=Tr(T_F^c)Tr(T_G^aT_G^b)\epsilon_{\mu\nu\rho\sigma}F_a^{\mu\nu}F_b^{\rho\sigma}\\
&=\frac{1}{16\pi^2}\sum q_iT(R_i)Tr_{fund}(F\widetilde{F})
\end{align}
\begin{equation}
\Delta Q_F=\int_{-\infty}^{\infty}dt\int d^3xj^0=\int d^4x\partial_\mu j^\mu=A_{FGG}\int d^4x F\widetilde{F}=A_{FGG}k
\end{equation}
Atiyah-Singer index theorem = (left-hand zero mode)-(right-hand zero mode)\\
\# zero modes in reprensentation $R$ is $T(R_i)$
\begin{equation}
L_{'t\ Hooft}\sim e^{-S_{inst}}\Pi_i\psi_i^{\tau(R_i)}
\end{equation}
FFF\\
't Hooft anomaly\\
(picture)\\
obstruction to gauging F\\
't Hooft: Such anomalies are const. along RG flow. Why cancel $A_{FFF}$ with spectators that don't affect G dynamics. Now weakly gauge F symmetry.
\begin{equation}
A_{FFF}^{orig}+A_{FFF}^{spect}=0
\end{equation}
for all RG scale $\mu$.\\
QCD symmetries\\
(table)\\
Since anomaly is proportional to $Tr(F\widetilde{F})$, which is $\theta$-term in action, can restore chiral invariance. If treat $\theta$ parameter as field and let it transform as
\begin{align}
\notag\psi^i&\rightarrow e^{iq_i\epsilon}\psi^i\\
\theta&\rightarrow\theta+\epsilon\sum_iq_iT(R_i)
\end{align}
Then
\begin{align}
\partial_\mu J^\mu&=\frac{\delta L}{\delta\epsilon}=\frac{\delta}{\delta\epsilon}(\frac{\theta}{16\pi^2}Tr_f(F\widetilde{F}))\\
\notag&=\mathrm{R.H.S.\ of\ anomaly}
\end{align}
In this way, chiral anomaly is now seen in classical action as explicit breaking.\\
Since anomaly appears only through $\theta$-terms, at most one anomalous U(1) per simple gauge.

Summary\\
Scale and chiral anomalies in gauge theories summarized in classical action by
\begin{equation}
S=\int d^4x[\frac{i\tau}{16\pi}Tr(F)^2+h.c.]
\end{equation}
with
\begin{equation}
\tau(\mu)=\frac{\theta}{2\pi}+\frac{4\pi i}{g^2(\mu)}=\frac{1}{2\pi i}\ln(\frac{\Lambda^b}{\mu^b})
\end{equation}
$\theta\rightarrow\theta+\epsilon A_{FGG}$ under U(1) chiral\\
$\Rightarrow\Lambda^b\rightarrow e^{i\epsilon A_{FGG}}\Lambda^b$,where
\begin{equation}
b=\frac{11}{6}T(adj)-\frac{1}{3}\sum_iT(R_i)-\frac{1}{6}\sum_aT(R_a)
\end{equation}
$$A_{FGG}=\sum_iq_iT(R_i)$$\\
In $\mathcal{N}=1$ superspace language, this becomes 
\begin{equation}
S=\int d^4x[\int d^2\theta\frac{1}{32\pi^2}\ln(\frac{\Lambda}{\mu})^bTrW^2+c.c.]+\mathrm{superpotential}+\mathrm{Kahler}
\end{equation}
Evaluate b, charge $\Lambda^b$\\
Say have $\Phi_i$ in $R_i$ representation on gauge group
\begin{equation}
\Phi_i\supset\{\phi_i,\psi_i\}
\end{equation}
where $\phi_i$ is a $\mathbb{C}$ scalar and $\psi_i$ is a Weyl scalar
\begin{equation}
W_\alpha\supset\{\lambda_\alpha,F_{\mu\nu}\}
\end{equation}
where $\lambda_\alpha$ is adj Weyl and $F_{\mu\nu}$ is the gauge  boson.
\begin{align}
b&=\frac{11}{6}T(adj)-\frac{1}{3}(T(adj)+\sum_iT(R_i))-\frac{1}{6}(\sum_iT(R_i))\\
&=\frac{1}{2}[3T(adj)-\sum_iT(R_i)]
\end{align}
What is $\Lambda^b$ charge: $\sum_{i:Weyl}q_iT(R_i)$, it depends on $q_i$. i.e. on definition of chiral U(1). Since there are many $U(1)$'s in free (massless) theory, there are many choices.\\
(1)Axial flavor $U(1)$'s\\
Symmetry under which only one chiral superfield rotate (clasical symmetry if there is no superpotential).
\begin{align}
\Phi&\rightarrow e^{i\epsilon}\Phi_i\qquad\psi_i \mathrm{has\ charge} 1\\
U(1)_i:=\Phi_j&\rightarrow\Phi_j\qquad\psi_j \mathrm{has\ charge} 0\quad(j\neq i)\\
W_\alpha&\rightarrow W-\alpha\qquad\lambda_\alpha \mathrm{has\ charge} 0
\end{align}
$$\Rightarrow\sum_{ferm}qT(R)=T(R_i)$$
So $\Lambda^b$ has $U(1)$: charge $T(R_i)$\\
(2)$U(1)_R$ anomaly\\
$$U(1)_R:\theta\rightarrow e^{i\epsilon}\theta \quad R[\theta]=1 \quad R[d\theta]=-1$$
For symmetry, we need $R[\tau TrW^2]=2$\\
As essentially convenient such R-symmetry is:\\
(table)

    
%\printbibliography
\end{document}
